\Chapter{Bevezetés}

Az automatizálás mindig is nagy szerepet játszott az emberiség életében, az információ korába belépve ennek fontossága magára az informatikára is kiterjed.

A dolgozat azt vizsgálja, hogy a számítógépek felhasználói felületén milyen sokszor elismételt folyamatok zajlanak le, ezek egy robot szempontjából hogyan is néznek ki, hogyan lehet ezeket utánozni, ezáltal automatizálni. Bemutatja a Robotikus Folyamat\hyp{}automatizálás (RPA) koncepcióját, annak egy speciális esetét, valamint az ahhoz kötő\hyp{}dő folyamatelemzési modellt.

Bár az RPA, mint fogalom annyira már nem új -- a 2000-es évek elején bukkant felszínre először --, mégsem annyira elterjedt a köztudatban. Bár sokaknak félelmetes lehet, rengeteg szempontból ezt a technológiát lehet tekinteni az emberi erőforrás gépiesítése felé tett egyik korai lépésnek.

Számos előnnyel rendelkezik a technológia, például olcsóbb egy cég számára, mint egy munkavállaló felvétele aki ugyanazokat a tevékenységeket csinálná. Ezen túl megbíz\hyp{}hatóbb is, hiszen az emberi munkással ellentétben nem unatkozik, nem fárad, így a hibalehetőségek aránya eredendően kissebb.
\newline\newline
Ezeken túllépve, a dolgozat inkább egy alacsonyabb szinten tekinti át a technológiát, azaz bemutatja egy gyakorlati implementációját, annak használatát, valamint a felépí\hyp{}tett folyamatokon elvégzett elemzéseket.

Ezzel már egy képet fog kapni az Olvasó arról, hogy a hétköznapokban előforduló folyamatokat hogyan tudja aránylag rövid időn belül automatizálni, teljes mértékben a saját kényel\hyp{}mének megfelelően.