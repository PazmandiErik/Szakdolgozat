\Chapter{Összefoglalás}

A dolgozat bemutatta, hogy mi is a Robotic Process Automation, annak elméletét, előnyeit, implementációját és gyakorlati alkalmazását. Felmérésre kerültek gyakran ismételt folyamatok, ezeket a dolgozathoz tartozó szoftver segítségével lehet generálni, szerkeszteni, illetve egy robot segítségével ismételni. Bemutatta a használt eszközöket, fejlesztői környezetet, valamint megindokolta annak kiválasztását.

Ezeken túl nagyvonalakban megismerhettük a folyamatelemzés elméletét és célját, valamint két modelljét; az Alpha-algoritmust és a Heurisztikus Bányászt. A dolgozat ezen részei kifejezetten nehéznek bizonyultak, mind az elméleti része, mind a gyakorlati megvalósítása. A folyamatmodell ábrázolása Petri-hálóként elég komplex műveletsor, talán ez az ami a leginkább időigényés volt.

Sikerült megvalósítani a terveket, illetve olyan funckiókat is, melyek eredetileg nem merültek fel.
A továbbiakban tervben van a szoftver továbbfejlesztése és optimalizálása, valamint az egyes részeinek teljeskörű átdolgozása. Ilyenek például a következők.
\begin{enumerate}
\item Az adatbányászat eredményeinek megjelenítése szebb formában történjen, kifeje\hyp{}zetten a Petri-hálót rajzoló algoritmust kell finomítani.
\item Az időzítő felületet teljesen át kell dolgozni, hogy felhasználóbarátabb legyen.
\item Érdemes egységesíteni a fájlformátumokat, mind a generált folyamatok, mind a kézzel letöltött folyamatok legyenek ugyanolyanok.
\item További folyamatelemzési algoritmusok implementálása.
\item Windows környezethez érdemes lehet service-ként újraírni a szoftver egyes részeit. Ez segítene bizonyos jogosultsági problémák megoldásában.
\end{enumerate}

Összességében a dolgozat eredményeképpen létrejött egy szoftver, melyet használni lehet folyamat\hyp{}automatizálásra, illetve folyamatelemzésre.