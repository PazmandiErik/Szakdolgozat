\pagestyle{empty}

\noindent \textbf{\Large Adathordozó használati útmutató}

\vskip 1cm

\noindent A szakdolgozathoz tartozó adathordozó a következőket tartalmazza:
\begin{enumerate}
\item \textbf{Szoftver futtatható állománya}

Elérési útvonal: \textit{root:/bin/Szakdolgozat.exe}

\item \textbf{Szoftver forráskódja}

Elérési útvonal: \textit{root:/src/}
\item \textbf{Szoftverhez tartozó kiegészítő állományok}

Elérési útvonal: \textit{root:/datamining/}

Bemutató elemezések találhatóak itt, használt előtt egy mappába ki kell csoma\hyp{}golni őket.
\item \textbf{Szakdolgozat dokumentum}

Elérési útvonal: \textit{root:/szakdolgozat.pdf}
\item \textbf{Szakdolgozat \LaTeX\ forrás}

Elérési útvonal: \textit{root:/doc/}
\end{enumerate}

A szoftver használatához az adathordozóról a számítógépre egy külön könyvtárba szükséges másolni a "Szakdolgozat.exe" fájlt. A folyamatokhoz tartozó bizonyos funkci\hyp{}ók (pl. időzítés) csak akkor működnek megfelelően, ha rendszergazdaként futtatjuk, hiszen erre a programra is érvényesek a Windows korlátozásai.